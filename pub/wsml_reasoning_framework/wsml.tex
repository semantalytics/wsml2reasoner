\section{Reasoning in the WSML Language\label{sec:wsml}}
-- give an overview of the WSML language; focus on semantics of rule-based variants (also show how syntax looks like, e.g. by an example) \\
-- focus on the ontology language part of WSML (only briefly mention WS-specific parts) and sketch its features, such as constraints, datatypes, conceptual modelling + axiomatic formulations, ... \\
-- describe the reasoning tasks in WSML, i.e. KB satisfiability and entailment \\
-- relate language features and reasoning, e.g. constraints and satisfiability, to emphasise the close connection of these features to reasoning \\

The Web Service Modeling Language (WSML) is a language for the
specification of different aspects of Semantic Web Services. It
provides a formal language for the Web Service Modeling Ontology
WSMO which is based on well-known logical formalisms, specifying one
coherent language framework for the semantic description of Web
Services, starting from the intersection of Datalog and the
Description Logic $\mathcal{SHIQ}$. This core language is extended
in the directions of Description Logics and Logic Programming in a
principled manner with strict layering. WSML distinguishes between
conceptual and logical modeling in order to support users who are
not familiar with formal logic, while not restricting the expressive
power of the language  for the expert user. IRIs play a central role
in WSML as identifiers. Furthermore, WSML defines XML and RDF
serializations for inter-operation over the Semantic Web.

The Web Service Modeling Language WSML takes into account all
aspects of Web Service description identified by WSMO, i.e. Web
services, goals, mediators and ontologies. WSML comprises different
formalisms in order to investigate their applicability to the
description of SWS. It would have been too restrictive to base our
effort on existing language recommendations such as OWL
\cite{Dean+Schreiber-OntoLangRefe:04}. A concrete goal in our
development of WSML is to investigate the usage of different
formalisms, most notably Description Logics and Logic Programming,
in the context of Ontologies and Web services. Within this paper we
furthermore focus only on the Ontology part of WSML. Note that we
expect that reasoning tasks with the other elements of WSMO, e.g.
the over capabilities is expected to be reducible to reasoning tasks
within ontologies.

WSML makes a clear distinction between the modeling of the different
conceptual elements on the one hand and the specification of complex
logical definitions on the other. To this end, the WSML syntax is
split into two parts: the conceptual syntax and logical expression
syntax. The conceptual syntax was developed from the user
perspective, and is independent from the particular underlying
logic; it shields the user from the peculiarities of the underlying
logic. Having such a conceptual syntax allows for easy adoption of
the language, since it allows for an intuitive understanding of the
language for people not familiar with logical languages. In case the
full power of the underlying logic is required, the logical
expression syntax can be used. There are several entry points for
logical expressions in the conceptual syntax, namely, axioms in
ontologies and capability descriptions in Goals and Web Services.

\subsection{Conceptual Syntax}
\label{sec:conceptual-syntax}

The WSML conceptual syntax for ontologies allows for the modeling of
concepts, instances, relations and relation instances. Additionally,
an ontology may have non-functional properties and may import other
ontologies. We start the description of WSML ontologies with an
example which demonstrates the elements of an ontology in Listing
\ref{lst:wsml-ontology-example}, and detail the elements below.

\begin{lstlisting}[label=lst:wsml-ontology-example,style=wsml,caption={An
    Example WSML Ontology}]
wsmlVariant _"http://www.wsmo.org/wsml/wsml-syntax/wsml-flight"
namespace {_"http://example.org/bookOntology#",
        dc _"http://purl.org/dc/elements/1.1/"}
ontology _"http://example.org/bookOntology"
  nonFunctionalProperties
    dc#title hasValue "Example Book ontology"
    dc#description hasValue "Example ontology about books and shopping carts"
  endNonFunctionalProperties
    concept book
        title ofType _string
        hasAuthor ofType author
    concept author subConceptOf person
        authorOf inverseOf(hasAuthor) ofType book
    concept cart
      nonFunctionalProperties
        dc#description hasValue "A shopping cart has exactly one id
          and zero or more items, which are books."
      endNonFunctionalProperties
        id ofType (1) _string
        items ofType book
    instance crimeAndPunishment memberOf book
        title hasValue "Crime and Punishment"
        hasAuthor hasValue dostoyevsky

    relation authorship(impliesType author, impliesType document)
      nonFunctionalProperties
        dc#relation hasValue authorshipFromAuthor
      endNonFunctionalProperties

    axiom authorshipFromAuthor
      definedBy
        authorship(?x,?y) :- ?x[authorOf hasValue ?y] memberOf author.
\end{lstlisting}

\paragraph{Concepts}
The notion of concepts (sometimes also called `classes') plays a
central role in ontologies. Concepts form the basic terminology of
the domain of discourse. A concept may have instances and may have a
number of attributes associated with it. The non-functional
properties, as well as the attribute definitions, are grouped
together in one frame, as can be seen from the example concept
\syn{book} in Listing \ref{lst:wsml-ontology-example}.

Attribute definitions can take two forms, namely \emph{constraining}
(using \synkw{ofType}) and \emph{inferring} (using
\synkw{impliesType}) attribute definitions\footnote{The distinction
  between inferring and constraining attribute definitions is
  explained in more detail in \cite[Section
  2]{Bruijn+PolleresETAL-:05}}. Constraining attribute definitions
define a typing constraint on the values for this attribute, similar
to integrity constraints in Databases; inferring attribute
definitions imply that the type of the values for the attribute is
inferred from the attribute definition, similar to range
restrictions on properties in
RDFS~\cite{Brickley+Guha-VocaDescLang:03} and
OWL~\cite{Dean+Schreiber-OntoLangRefe:04}.  Each attribute
definition may have a number of features associated with it, namely,
transitivity, symmetry, reflexivity, and the inverse of an
attribute, as well as minimal and maximal cardinality constraints.

Constraining attribute definitions, as well as cardinality
constraints, require closed-world reasoning and are thus not allowed
in WSML-Core and WSML-DL. As opposed to features of roles in
Description Logics, attribute features such as transitivity,
symmetry, reflexivity and inverse attributes are local to a concept
in WSML. Thus, none of these features may be used in WSML-Core and
WSML-DL. For a motivation on the use of constraining attributes, see
\cite{Bruijn+PolleresETAL-:05}.

\paragraph{Relations}
Relations in WSML can have an arbitrary arity, may be organized in a
hierarchy using \synkw{subRelationOf} and the parameters may be
typed using parameter type definitions of the form
(\syn{\synkw{ofType}
  $type$} ) and (\syn{\synkw{impliesType} $type$}), where $type$ is a
concept identifier. The usage of \synkw{ofType} and
\synkw{impliesType} correspond with the usage in attribute
definitions. Namely, parameter definitions with the \synkw{ofType}
keyword are used to check the type of parameter values, whereas
parameter definitions with the \synkw{impliesType} keyword are used
to infer concept membership of parameter values.

The allowed arity of the relation may be constrained by the
underlying logic of the WSML language variant. WSML-Core and WSML-DL
allow only binary relations and, similar to attribute definitions,
they allow only parameter typing using the keyword
\synkw{impliesType}.


\paragraph{Instances}
A concept may have a number of instances associated with it.
Instances explicitly specified in an ontology are those which are
shared as part of the ontology. However, most instance data exists
outside the ontology in private databases. WSML does not prescribe
how to connect such a database to an ontology, since different
organizations will use the same ontology to query different
databases and such corporate databases are typically not shared.

An instance may be member of zero or more concepts and may have a
number of attribute values associated with it, see for example the
instance \syn{crimeAndPunishment} in Listing
\ref{lst:wsml-ontology-example}. Note that the specification of
concept membership is optional and the attributes used in the
instance specification do not necessarily have to occur in the
associated concept definition. Consequently, WSML instances can be
used to represent semi-structured data, since without concept
membership and constraints on the use of attributes, instances form
a directed labelled graph. Because of this possibility to capture
semi-structured data, most RDF graphs can be represented as WSML
instance data, and vice versa.


\paragraph{Axioms} Axioms provide a means to add arbitrary logical
expressions to an ontology. Such logical expressions can be used to
refine concept or relation definitions in the ontology, but also to
add arbitrary axiomatic domain knowledge or express constraints. The
axiom \syn{authorshipFromAuthor} in Listing
\ref{lst:wsml-ontology-example} states that the relation
\syn{authorship} exists between any author and any book of which he
is an author; consequently, $\langle \syn{dostoyesksy},
\syn{crimeAndPunishment} \rangle$ is in the relation
\syn{authorship}. Logical expressions are explained in more detail
in Section~\ref{sec:log-expr-syntax}.

\subsection{Logical Expression Syntax}
\label{sec:log-expr-syntax}

We will first explain the general logical expression syntax, which
encompasses all WSML variants, and then describe the restrictions on
this general syntax for each of the variants. The general logical
expression syntax for WSML has a First-Order Logic style, in the
sense that it has constants, function symbols, variables, predicates
and the usual logical connectives. Furthermore, WSML has F-Logic
\cite{Kifer+LausenETAL-LogiFounObjeFram:95} based extensions in
order to model concepts, attributes, attribute definitions, and
subconcept and concept membership relationships. Finally, WSML has a
number of connectives to facilitate the Logic Programming based
variants, namely default negation (negation-as-failure),
LP-implication (which differs from classical implication) and
database-style integrity constraints.

Variables in WSML start with a question mark, followed by an
arbitrary number of alphanumeric characters, e.g., \syn{?x},
\syn{?name}, \syn{?123}. Free variables in WSML (i.e., variables
which are not explicitly quantified), are implicitly universally
quantified outside of the formula (i.e., the logical expression in
which the variable occurs is the scope of quantification), unless
indicated otherwise, through the \synkw{sharedVariables} construct
(see the previous Section).

Terms are either identifiers, variables, or constructed terms. An
atom is, as usual, a predicate symbol with a number of terms as
arguments. Besides the usual atoms, WSML has a special kind of
atoms, called \emph{molecules}, which are used to capture
information about concepts, instances, attributes and attribute
values. The are two types of molecules, analogous to F-Logic:

\begin{itemize}
\item An \emph{isa} molecule is a concept membership molecule of
  the form \syn{\emph{A} \synkw{memberOf} \emph{B}} or a subconcept
  molecule of the form \syn{\emph{A} \synkw{subConceptOf} \emph{B}}
  with \syn{\emph{A}} and \syn{\emph{B}} arbitrary terms

\item An \emph{object} molecule is an attribute value expressions
  of the form \syn{\emph{A}[\emph{B} \synkw{hasValue} \emph{C}]}, a
  constraining attribute signature expression of the form
  \syn{\emph{A}[\emph{B} \synkw{ofType} \emph{C}]}, or an inferring attribute signature expression of the form
  \syn{\emph{A}[\emph{B} \synkw{ofType} \emph{C}]}, with
  \syn{\emph{A,B,C}} arbitrary terms
\end{itemize}

WSML has the usual first-order connectives: the unary negation
operator \synkw{neg}, and the binary operators for conjunction
\synkw{and}, disjunction \synkw{or}, right implication
\synkw{implies}, left implication \synkw{impliedBy}, and dual
implication \synkw{equivalent}. Variables may be universally
quantified using \synkw{forall} or existentially quantified using
\synkw{exists}. First-order formulae are obtained by combining atoms
using the mentioned connectives in the usual way. The following are
examples of First-Order formulae in WSML:

\begin{lstlisting}[style=wsml, frame=none]
//every person has a father
forall ?x (?x memberOf Person implies exists ?y (?x[father hasValue
?y])).
//john is member of a class which has some attribute called 'name'
exists ?x,?y (john memberOf ?x and ?x[name ofType ?y]).
\end{lstlisting}

Apart from First-Order formulae, WSML allows the use of the
negation-as-failure symbol \synkw{naf} on atoms, the special Logic
Programming implication symbol \synkw{:-} and the integrity
constraint symbol \synkw{!-}. A logic programming rule consists of a
\emph{head} and a \emph{body}, separated by the \synkw{:-} symbol.
An integrity constraint consists of the symbol \synkw{!-} followed
by a rule body. Negation-as-failure \synkw{naf} is only allowed to
occur in the body of a Logic Programming rule or an integrity
constraint. The further use of logical connectives in Logic
Programming rules is restricted. The following logical connectives
are allowed in the head of a rule: \synkw{and}, \synkw{implies},
\synkw{impliedBy}, and \synkw{equivalent}. The following connectives
are allowed in the body of a rule (or constraint): \synkw{and},
\synkw{or}, and \synkw{naf}. The following are examples of LP rules
and database constraints:

\begin{lstlisting}[style=wsml, frame=none]
//every person has a father
?x[father hasValue f(?y)] :- ?x memberOf Person.
//Man and Woman are disjoint
!- ?x memberOf Man and ?x memberOf Woman.
//in case a person is not involved in a marriage, the person is a bachelor
?x memberOf Bachelor :- ?x memberOf Person and naf
Marriage(?x,?y,?z).
\end{lstlisting}

\subsection{Particularities of the WSML Variants}

Each of the WSML variants defines a number of restrictions on the
logical expression syntax. For example, LP rules and constraints are
not allowed in WSML-Core and WSML-DL. Table \ref{table:wsml-matrix}
presents a number of language features and indicates in which
variant the feature can occur.

\begin{table}[ht]
\center
\begin{tabular}{|l|c|c|c|c|c|}
    \hline
Feature             & Core  & DL    & Flight    & Rule  & Full \\
    \hline
Classical Negation (\synkw{neg})
                    & -     & X     & -         & -     & X \\
Existential Quantification
                    & -     & X     & -         & -     & X \\
(Head) Disjunction
                    & -     & X     & -         & -     & X \\
$n$-ary relations
                    & -     & -     & X         & X     & X \\
Meta Modeling
                    & -     & -     & X         & X     & X \\
Default Negation (\synkw{naf})
                    & -     & -     & X         & X     & X \\
LP implication
                    & -     & -     & X         & X     & X \\
Integrity Constraints
                    & -     & -     & X         & X     & X \\
Function Symbols
                    & -     & -     & -         & X     & X \\
Unsafe Rules
                    & -     & -     & -         & X     & X \\
    \hline
\end{tabular}
\caption{WSML Variants and Feature Matrix} \label{table:wsml-matrix}
\end{table}

\begin{itemize}
\item \emph{WSML-Core}
allows only first-order formulae which can be translated to the DLP
subset of $\mathcal{SHIQ}(\mathbf{D)}$
\cite{Grosof+HorrocksETAL-DescLogiProg:03}. This subset is very
close to the 2-variable fragment of First-Order Logic, restricted to
Horn logic. Although WSML-Core might appear in the Table
\ref{table:wsml-matrix} featureless, it captures most of the
conceptual model of WSML, but has only limited expressiveness within
the logical expressions.

\item \emph{WSML-DL}
allows first-order formulae which can be translated to
$\mathcal{SHIQ}(\mathbf{D)}$. This subset is very close to the
2-variable fragment of First-Order Logic. Thus, WSML DL allows
classical negation, and disjunction and existential quantification
in the heads of implications.

\item \emph{WSML-Flight}
extends the set of formulae allowed in WSML-Core by allowing
variables in place of instance, concept and attribute identifiers
and by allowing relations of arbitrary arity. In fact, any such
formula is allowed in the head of a WSML-Flight rule. The body of a
WSML-Flight rule allows conjunction, disjunction and default
negation. The head and body are separated by the LP implication
symbol.

WSML-Flight additionally allows meta-modeling (e.g.,
classes-as-instances) and reasoning over the signature, because
variables are allowed to occur in place of concept and attribute
names.

\item \emph{WSML-Rule}
extends WSML-Flight by allowing function symbols and unsafe rules,
i.e., variables which occur in the head or in a negative body
literal do not need to occur in a positive body literal.

\item \emph{WSML-Full}
The logical syntax of WSML-Full is equivalent to the general logical
expression syntax of WSML and allows the full expressiveness of all
other WSML variants.

\end{itemize}

The separation between conceptual and logical modeling allows for an
easy adoption by non-experts, since the conceptual syntax does not
require expert knowledge in logical modeling, whereas complex
logical expressions require more familiarity and training with the
language. Thus, WSML allows the modeling of different aspects
related to Web services on a conceptual level, while still offering
the full expressive power of the logic underlying the chosen WSML
variant. Part of the conceptual syntax for ontologies has an
equivalent in the logical syntax. This correspondence is used to
define the semantics of the conceptual syntax. Notice that, since
only parts of the conceptual syntax are mapped to the logical
syntax, only a part of the conceptual syntax has a semantics in the
logical language for ontologies. For example, non-functional
properties are not translated (hence, the name `non-functional').
The translation between the conceptual and logical syntax is
sketched in Table \ref{tab:conceptual-logical}.

\begin{table}[ht]
  \centering
\footnotesize{
  \begin{tabular}{l|l}
    \textbf{Conceptual} & \textbf{Logical} \\
\hline
    \begin{lstlisting}[style=wsml-table]
concept A subConcepOf B
    \end{lstlisting}
    &
    \begin{lstlisting}[style=wsml-table]
A subConceptOf B.
    \end{lstlisting}
    \\
    \begin{lstlisting}[style=wsml-table]
concept A
  B ofType (0 1) C
    \end{lstlisting}
    &
    \begin{lstlisting}[style=wsml-table]
A[B ofType C]. !- ?x memberOf A and
  ?x[B hasValue ?y, B hasValue ?z] and ?y != ?z.
    \end{lstlisting}\\
    \begin{lstlisting}[style=wsml-table]
concept A B ofType C
    \end{lstlisting}
    &
    \begin{lstlisting}[style=wsml-table]
A[B ofType C].
    \end{lstlisting}\\
    \begin{lstlisting}[style=wsml-table]
relation A/n subRelationOf B
    \end{lstlisting}
    &
    \begin{lstlisting}[style=wsml-table]
A(x$_1$,...,x$_n$) implies B(x$_1$,...,x$_n$)
    \end{lstlisting}\\
    \begin{lstlisting}[style=wsml-table]
instance A memberOf B
  C hasValue D
    \end{lstlisting}
    &
    \begin{lstlisting}[style=wsml-table]
A memberOf B. A[C hasValue D].
    \end{lstlisting}\\
    \hline
  \end{tabular}}

  \caption{Translating conceptual to logical
    syntax}\label{tab:conceptual-logical}
\end{table}
