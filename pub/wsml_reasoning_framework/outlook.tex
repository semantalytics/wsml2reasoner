\section{Conclusion \& Outlook\label{sec:outlook}}
In this paper, we presented a transformational framework that
enables us to perform important reasoning tasks for rule-based
WSML. The key features of the framework are: (1) Reasoning via
transformation to the widely used Datalog formalism with numerous
implemented systems (2) Modular structure of the transformation
allows for reuse of single well-defined transformation steps when
adapting and extending the transformation for different scenarios
(e.g. support for debugging of ontologies) (3) Simple integration
and exchange of underlying reasoning components. This allows to
customize the framework for specific applications: Application
developers can choose to use their desired reasoning system
providing the capabilities they need. If an application only uses
a less expressive WSML variant, then a more specialized reasoning
component tailored towards this language can be used. Such a
decision can even be changed during runtime (4) Support for
debugging of ontologies independent of the underlying reasoning
system. This feature can be dynamically added and removed from the
reasoning component.

 Eventually, the available implementation of the
system based on Datalog components such as KAON2 and MINS
represent the first available reasoning system for WSML. The
presented framework proved to be a flexible and effective way to
build reasoners for WSML based on existing, well-designed and
efficient systems in a short period of time.

Currently, the framework as well as our implementation focuses on
WSML Core, Flight and Rule. However, efforts are ongoing to extend
it into the direction of WSML DL and WSML Full.
%
%The following extensions to the framework are desirable:
%
%Another candidate for extension of language support is DLV
%(disjunctive datalog with default negation (under stable model
%semantics, which is fine for the stratified case), but poor in
%datatypes) and KRHyper (disjunctive logic programs with default
%negation in the stratified case). Hence DLV could be used for
%supporting WSML-DL reasoning in the precense of WSML-Flight rules
%whereas KRHyper could be used to reason with WSML-DL extended by
%WSML-Rule rules in the stratified case.
%
%The generic and flexible architecture of the framework, allows.
%
%
%-- \dots
%
%-- Nathalies stuff on DL


\paragraph{Acknowledgements.}
This work is supported by the European Commision under the the DIP
project (FP6-507483), and by the Austrian Federal Ministry for
Transport, Innovation, and Technology under the project {\sffamily
 {\bfseries R}W$^{\mathsf{2}}$} (FFG 809250).
