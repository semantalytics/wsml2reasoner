\section{Conclusion \& Outlook\label{sec:outlook}}
In this paper, we presented a transformational framework that
enables us to perform important reasoning tasks for rule-based
WSML. The key features of the framework are: (1) Reasoning via
transformation to the widely used Datalog formalism with numerous
implemented systems (2) Modular structure of the transformation
allows for adaptation and extension of the overall transformation.
The single well-defined transformation steps can be reused across
various adaptations for different scenarios (e.g. support for
debugging of ontologies) (3) Simple integration and exchange of
underlying reasoning components. This allows to customize the
framework for specific applications: Application developers can
choose to use their desired reasoning system providing the
capabilities they need. If an application only uses a less
expressive WSML variant, then a more specialized reasoning
component tailored towards this language can be used, e.g. to
ensure performance and scalability. Such a decision can even be
change during runtime (4) Support for debugging of ontologies
independent of the underlying reasoning system. This feature can
be dynamically added and removed from \emph{any} reasoning
component.

 Eventually, the available implementation of the
system based on Datalog components such as KAON2 and MINS
represent the first available reasoning system for WSML. The
presented framework proved to be a flexible and effective way to
build reasoners for WSML based on existing, well-designed and
efficient systems in a short period of time.

Currently, the framework as well as our implementation focuses on
WSML Core, Flight and Rule. However, efforts are ongoing to extend
this into the direction of WSML DL and WSML Full: we are working
on extending the transformations to allow for transformations to
disjunctive datalog and disjunctive logic programs (including
default negation) too. The KAON2 natively system already supports
disjunctive datalog with stratified default negation and thus can
be used for reasoning with WSML DL ontologies, even if they are
extended by WSML-Flight-like rules. The DLV
system~\cite{citrigno97dlv} (implementing disjunctive datalog
under the stable model semantics) can be used for reasoning the
same purpose. Furthermore, we plan to integrate the KRHyper
system~\cite{wernhard03system}, which allows reasoning with
disjunctive logic programs with stratified default negation. This
would then allow to reason with WSML DL, WSML Core, WSML Flight
and WSML Rule (and combinations) in an integrate manner in the
case of stratified negation and safe rules. This way, we expect to
be able to support significant parts of WSML Full. An additional
translation to widely used description logic (DL) system APIs
(e.g. DIG~\cite{BechhoferMC03}) to support efficient reasoning
with WSML-DL based on state-of-the-art DL systems like
Racer~\cite{haarslev01racer}, Pellet\footnote{\tt
http://www.mindswap.org/2003/pellet/} or Fact++\footnote{\tt
http://owl.man.ac.uk/factplusplus/} is on the way.

\paragraph{Acknowledgements.}
This work is supported by the European Commision under the the DIP
project (FP6-507483), and by the Austrian Federal Ministry for
Transport, Innovation, and Technology under the project {\sffamily
 {\bfseries R}W$^{\mathsf{2}}$} (FFG 809250).
