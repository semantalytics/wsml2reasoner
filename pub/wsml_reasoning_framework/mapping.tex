\newcommand{\smtxtit}[1]{\begin{scriptsize}\ensuremath{\textit{#1}}\end{scriptsize}}
\newcommand{\trans}[1]{\ensuremath{\tau_{#1}}\xspace}
\newcommand{\transtxt}[1]{\trans{\smtxtit{#1}}}
%\newcommand{\transmap}[3]{\ensuremath{\tau_{#1}^{{#2},{#3}}}\xspace}
%\newcommand{\transtxtmap}[3]{\transmap{\smtxtit{#1}}{#2}{#3}}
\def\LE{\ensuremath{\mathcal{L\!E}}\xspace}
\def\O{\ensuremath{\mathcal{O}}\xspace}
\def\P{\ensuremath{\mathcal{P}}\xspace}
\newcommand{\powset}[1]{\ensuremath{2^{#1}}\xspace}

\section{Realistion of WSML Reasoning through a Mapping to Datalog\label{sec:mapping}}
-- briefly sketch the idea of reasoning via rule based inferencing \\

The semantics of rule-based WSML is defined via a mapping to
datalog with (in)equality and integrity constraints. \sgr{probably
use a special denotation like $\textit{datalog}^{!=,IC}$, which
then has to be introduced in Section 2} Thus, the reasoning
framework performs various transformations to convert an original
ontology in WSML syntax into datalog rules. To maintain the
semantics of more complex WSML language constructs that cannot
directly be expressed in datalog, a fixed set of rules form the
meta-level axioms that realise part of the WSML semantics during
reasoning. Finally, the WSML reasoning tasks of knowledge base
satisfiability and instance retrieval are realised by datalog
querying via calls to an underlying datalog inference engine that
is fed with the rules produced during transformation together with
the meta-level axioms.

\subsection{Transformation of WSML into Datalog}
-- describe different transformation steps \\

The transformation of a WSML ontology to datalog rules forms a
pipeline of single transformation steps which are subsequently
applied, starting from the original ontology.

\paragraph{Axiomatization} -- In a first step, the transformation
\transtxt{axiom} is applied as a mapping $\O \mapsto \powset{\LE}$
from the set of all valid ontologies formulated in rule-based WSML
to the powerset of all logical expressions that conform to
rule-based WSML. \transtxt{axiom} converts all conceptual syntax
elements, such as concept and attribute definitions or cardinality
and type constraints, into appropriate logical expressions
according to \cite{wsml-spec}(Table 8.1). \sgr{give the complete
conversion table ??}

To give an example, the WSML fragment
\begin{quote}
\begin{lstlisting}[style=wsml]
concept C subConceptOf D
    r ofType (0 2) T
instance a memberOf C
    r hasValue b,c
\end{lstlisting}
\end{quote}
is translated by \transtxt{axiom} to the following logical
expressions.

\begin{quote}
\begin{lstlisting}[style=wsml]
C subConceptOf D.%
!- ?x memberOf C and ?x[r hasValue ?y] and naf ?y memberOf T.%
!- ?x memberOf C and ?x[r hasValue ?y1, r hasValue ?y2] and ?y1 != ?y2.%
a memberOf C.  a hasValue b,c.
\end{lstlisting}
\end{quote}

\paragraph{Normalization} -- The transformation \transtxt{norm} is
applied to normalize WSML logical expressions as a mapping
$\powset{\LE} \mapsto \powset{\LE}$. This normalization step
reduces the complexity of WSML logical expressions according to
\cite{wsml-spec}(Section 8.2) to make them better fit the simple
syntactic form of literals in datalog rules. This reduction
includes conversion to negation and disjunctive normal forms as
well as decomposition of complex WSML molecules (see
\cite{wsml-spec}).

\paragraph{Lloyd-Topor Transformation} -- The transformation
\transtxt{Lloyd-Topor} is applied as a mapping $\powset{\LE}
\mapsto \powset{\LE}$ to flatten the complex WSML logical
expressions, producing simple rules according to the Lloyd-Topor
transformations \cite{lloyd-topor} as follows.
\begin{center}
\begin{tabular}{|c|c|}
  \hline
  % after \\: \hline or \cline{col1-col2} \cline{col3-col4} ...
  \emph{original expression} & \emph{simplified rule(s)} \\
  \hline
  $H_1 \wedge H_2 \wedge \dots \wedge H_n \leftarrow B$ & $H_1 \leftarrow B , H_2 \leftarrow B , \dots , H_n \leftarrow B$ \\
  $H_1 \leftarrow H_2 \leftarrow B$ & $H_1 \leftarrow H_2 \wedge B$ \\
  $H \leftarrow B_1 \vee B_2 \vee \dots \vee B_n$ & $H \leftarrow B_1 , H \leftarrow B_2 , \dots , H \leftarrow B_n$ \\
  \hline
\end{tabular}
\end{center}

After this step, the resulting WSML expressions have the form of
proper datalog rules with a single head and conjunctive (possibly
negated) body literals.

\paragraph{Datalog Rule Generation} -- In a final step, the
transformation \transtxt{datalog} is applied as a mapping
$\powset{\LE} \mapsto \P$ from all valid logical expressions in
rule-based WSML to the set of all datalog programs, yielding
generic datalog rules that represent the content of the original
WSML ontology. In this generic datalog program, all remaining
WSML-specific language constructs, such as \wsml{subConceptOf} or
\wsml{ofType}, are replaced by special meta-level predicates for
which the semantics of the respective language construct is
encoded in meta-level axioms as described in a further subsection.
\begin{table}[b]\label{tab:LE2datalog}\centering
\begin{tabular}{|l|c|}
  \hline
  bla & bla \\
  \hline
\end{tabular}
\caption{Transformation from logical expressions in rule-based
WSML to datalog including meta-level predicates.}
\end{table}
Table \ref{tab:LE2datalog} shows the mapping from WSML logical
expressions to generic datalog including such meta-level
predicates.

Notice that the \wsml{memberOf} construct for instantiation is
also mapped to an appropriate meta-level predicate instead of
direct instantiation of the form $C(i)$, which is available in
datalog. This decision was taken in order to facilitate the
metamodelling capabilities of rule-based WSML, which allows an
entity to be both an instance and class at the same time.
\sgr{genauer darstellen f�r was welche metal-level Predikate
erzeugt werden!}

\bigskip

Ultimately, the basic transformation pipeline for converting a
rule-based WSML ontology into a datalog program is the following,
constituted by the single transformation steps introduced before.
\begin{displaymath}
    \tau = \transtxt{datalog} \circ \transtxt{Lloyd-Topor} \circ \transtxt{norm} \circ \transtxt{axiom}
\end{displaymath}
As a mapping $\O \mapsto \P$, this chaining of the single steps is
applied to a WSML ontology $O \in \O$ to yield a semantically
equivalent datalog program $\tau (O) = P \in \P$ when interpreted
with respect to the meta-level axioms discussed next.

\subsection{Realising WSML Semantics through Meta Axioms}
-- describe how a fixed set of rules implements (part of) the WSML semantics during reasoning \\

The mapping from WSML to datalog in the reasoning framework works
such that all WSMl-identifiable entities, namely concepts,
instances, attributes etc., are mapped to datalog instances.

\subsection{Mapping WSML Reasoning Tasks to Datalog Querying}
-- describe how to realise WSML satisfiability and entailment to datalog querying \\

\subsection{Realising Datatype Reasoning}
-- describe how reasoning with datatypes is realised
