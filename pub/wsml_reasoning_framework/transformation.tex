% meta-variables.
\def\mvexpr{\ensuremath{E}}
\def\mvex{\ensuremath{E_x}}
\def\mvey{\ensuremath{E_y}}
\def\mvez{\ensuremath{E_z}}
\def\mve1{\ensuremath{E_1}}
\def\mven{\ensuremath{E_n}}
%\def\mvhd{\ensuremath{H}}
%\def\mvh1{\ensuremath{H_1}}
%\def\mvh2{\ensuremath{H_2}}
%\def\mvhn{\ensuremath{H_n}}
%\def\mvbd{\ensuremath{b}}
%\def\mvb1{\ensuremath{B_1}}
%\def\mvbn{\ensuremath{B_n}}

\subsection{Ontology Transformations}
The transformation of a WSML ontology to Datalog rules forms a
pipeline of single transformation steps that are subsequently
applied, starting from the original ontology.\\[2mm]
{\it Axiomatization.} In a first step, the transformation \transax
is applied as a mapping $\O \rightarrow \powset{\LE}$ from the set
of all valid rule-based WSML ontologies to the powerset of all
logical expressions that conform to rule-based WSML. In this
transformation step, all conceptual syntax elements, such as concept
and attribute definitions or cardinality and type constraints, are
converted into appropriate axioms specified by logical expressions.
Table \ref{tab:axiomatization} shows the details of some of the
conversions performed by \transax, based on \cite{wsml-spec}. The
WSML conceptual syntax constructs on the left-hand side are
converted to the respective WSML logical expressions on the
right-hand side. The meta variables $C,C_i$ range over identifiers
of WSML concepts, $R_i,A_i$ over identifiers of WSML relations and
attributes, $T$ over identifiers of WSML concepts or datatypes and
$V$ over identifiers of WSML instances or data values.
\begin{table}[tbp]
\centering
\begin{scriptsize}
\begin{tabular}{|l|l|}
  \hline
  \rule{0cm}{3.2mm}{\small \emph{Expression $\alpha$ in conceptual syntax}} & {\small \emph{Resulting logical expression(s)}: $\transax(\alpha)$}  \\
  \hline
    \twsml{concept} $C_1$ \twsml{subConceptOf} $C_2$  & $C_1$ \twsml{subConceptOf} $C_2.$ \\
  \hline
      \twsml{concept} $C$ $A$ \twsml{ofType} $(0,1)$ $T$  & $C$[A \twsml{ofType} $T$]. \\[-0.5ex]  &
      !- ?x \twsml{memberOf} $C$ \twsml{and} ?x[A \twsml{hasValue} ?y, A \twsml{hasValue} ?z] \twsml{and} ?y != ?z. \\
  \hline
      \twsml{concept} $C$ A$_1$ \twsml{inverseOf} A$_2$ \twsml{impliesType} $T$ & $C$[A \twsml{impliesType} $T$]. \\[-0.5ex]
      & ?x \twsml{memberOf} $C$ \twsml{and} ?v \twsml{memberOf} $T$ \twsml{implies} \\[-0.5ex]
      & \enspace ( ?x[A$_1$ \twsml{hasValue} ?v] \twsml{equivalent} ?v[A$_2$ \twsml{hasValue} ?x] ). \\
  \hline
      \twsml{relation} $R_1$/$n$ \twsml{subRelationOf} $R_2$ & $R_1(\vec{x})$ \twsml{implies} $R_2(\vec{x})$.  \begin{scriptsize}where\end{scriptsize} $\vec{x}$ = (x$_1$,...,x$_n$) \\
  \hline
      \twsml{instance} $I$ \twsml{memberOf} $C$ A \twsml{hasValue} $V$ & $I$ \twsml{memberOf} $C$. I[A \twsml{hasValue} $V$]. \\
\hline
\end{tabular}
\end{scriptsize}
\caption{Examples for axiomatizing conceptual ontology modeling
elements.} \label{tab:axiomatization}
\end{table}
\\[2mm]
{\it Normalization.} The transformation \transnorm is applied as a
mapping $\powset{\LE} \rightarrow \powset{\LE}$ to normalize WSML
logical expressions. This normalization step reduces the complexity
of formulae according to \cite[Section 8.2]{wsml-spec}, to bring
expressions closer to the simple syntactic form of literals in
Datalog rules. The reduction includes conversion to negation and
disjunctive normal forms as well as decomposition of complex WSML
molecules. The left part of Table \ref{tab:normalization} shows how
the various logical expressions are normalized in detail. The meta
variables $E_i$ range over logical expressions in rule-based WSML,
while $X,Y_i$ range over parts of WSML molecules. After \transnorm
has been applied, the resulting expressions have the form of logic
programming rules with no deep nesting of logical connectives.
%
\def\smalldots{\begin{tiny}\dots\end{tiny}}
\renewcommand{\baselinestretch}{1}
\begin{table}[tbp]\centering
\begin{scriptsize}
\begin{tabular}{|l|l|||l|l|}
  \hline
  \rule{0cm}{3.2mm}{\small \emph{original expression}} & {\small \emph{normalized expression}} & {\small \emph{original expression}} & {\small \emph{simplified rule(s)}}  \\
  \hline
    $\transnorm(\{\mve1 , \smalldots , \mven\})$ & $\{\transnorm(\mve1) , \smalldots , \transnorm(\mven)\}$ & $\transdlog(\{\mve1, \smalldots , \mven\})$ & $\{\transdlog(\mve1), \smalldots , \transdlog(\mven)\}$ \\
    $\transnorm(\mvex$ \twsml{and} $\mvey.)$ & $\transnorm(\mvex)$ \twsml{and} $\transnorm(\mvey)$ & $\transdlog($ \twsml{\cstr} $B.)$ & $\dlogcstr \transdlog(B)$ \\
    $\transnorm(\mvex$ \twsml{or} $\mvey.)$ & $\transnorm(\mvex)$ \twsml{or} $\transnorm(\mvey)$ & $\transdlog(H.)$ & \dlogfact{\transdlog(H)} \\
    $\transnorm(\mvex$ \twsml{and} $(\mvey$ \twsml{or} $\mvez).)$ & $\transnorm(\transnorm(\mvex)$ \twsml{and} $\transnorm(\mvey)$ \twsml{or} & $\transdlog(H$ \twsml{\lprl} $B.)$ & $\transdlog(H) \dlogrule \transdlog(B)$ \\
    & $\phantom{\transnorm(}\transnorm(\mvex)$ \twsml{and} $\transnorm(\mvez).)$ & $\transdlog(\mvex$ \twsml{and} $\mvey.)$ & $\transdlog(\mvex) \dlogand \transdlog(\mvey)$ \\
    $\transnorm((\mvex$ \twsml{or} $\mvey)$ \twsml{and} $\mvez).)$ & $\transnorm(\transnorm(\mvex)$ \twsml{and} $\transnorm(\mvez)$ \twsml{or} & $\transdlog($\twsml{naf} $\mvexpr.)$ & $\dlognot \transdlog(\mvexpr)$ \\
    & $\phantom{\transnorm(}\transnorm(\mvey)$ \twsml{and} $\transnorm(\mvez).)$ & $\transdlog(C_x$ \twsml{subConceptOf} $C_y.)$ & $\psco(C_x,C_y)$ \\
    $\transnorm($ \twsml{naf} $ (\mvex$ \twsml{and} $\mvey).)$ & $$ \twsml{naf} $ \transnorm(\mvex)$ \twsml{or} $$ \twsml{naf} $ \transnorm(\mvey).$ & $\transdlog(I$ \twsml{memberOf} $C.)$ & $\pmo(I,C)$ \\
    $\transnorm($ \twsml{naf} $ (\mvex$ \twsml{or} $\mvey).)$ & $$ \twsml{naf} $ \transnorm(\mvex)$ \twsml{and} $$ \twsml{naf} $ \transnorm(\mvey).$ & $\transdlog(I[a$ \twsml{hasValue} $V].)$ & $\phval(I,a,V)$ \\
    $\transnorm($ \twsml{naf} $ ($ \twsml{naf} $ \mvex).)$ & $\transnorm(\mvex)$ & $\transdlog(C[a$ \twsml{impliesType} $T].)$ & $\pitype(C,a,T)$ \\
    $\transnorm(\mvex$ \twsml{implies} $\mvey.)$ & $\transnorm(\mvey)$\twsml{\lprl}$\transnorm(\mvex).$ & $\transdlog(C[a$ \twsml{ofType} $T].)$ & $\potype(C,a,T)$ \\
    $\transnorm(\mvex$ \twsml{impliedBy} $\mvey.)$ & $\transnorm(\mvex)$\twsml{\lprl}$\transnorm(\mvey).$ & $\transdlog($\twsml{r}$(X_1, \smalldots , X_n).)$ & $r(X_1, \smalldots , X_n)$ \\
    $\transnorm(X[Y_1 , \smalldots , Y_n].)$ & $X[Y_1]$ \twsml{and} $\smalldots$ \twsml{and} $X[Y_n].$ & $\transdlog(X$ \twsml{=} $Y.)$ & $X = Y$ \\
    & & $\transdlog(X$ \twsml{!=} $Y.)$ & $X \neq Y$ \\
%  \hline
%  \hline
%  \rule{0cm}{3.2mm}{\small \emph{original expression}} & {\small \emph{simplified rule(s)}} & {\small \emph{original expression}} & {\small \emph{simplified rule(s)}} \\
%  \hline
%  $\translt( \{ \mve1 , \smalldots , \mven \})$ & $\{ \translt(\mve1) , \smalldots , \translt(\mven) \}$ & $\translt(H_1$ \twsml{and} $\smalldots$ \twsml{and} $H_n$\twsml{\lprl}$B.)$ & $\translt(H_1$\twsml{\lprl}$B.)$ , \smalldots , $\translt(H_n$\twsml{\lprl}$B.)$ \\
%  $\translt(H_1$\twsml{\lprl}$H_2$\twsml{\lprl}$B.)$ & $\translt(H_1$\twsml{\lprl}$H_2$ \twsml{and} $B.)$ & $\translt(H$\twsml{\lprl} $B_1$ \twsml{or} , $\smalldots$ , \twsml{or} $B_n.)$ & $\translt(H$\twsml{\lprl}$B_1.)$ , \smalldots , $\translt(H$\twsml{\lprl}$B_n.)$ \\
  \hline
\end{tabular}
\begin{tabular}{|l|l||l|l|}
  \hline
  \rule{0cm}{3.2mm}{\small \emph{original expr.}} & {\small \emph{simplified rule(s)}} & {\small \emph{original expression}} & {\small \emph{simplified rule(s)}} \\
  \hline
  $\translt( \{ \mve1 , \smalldots , \mven \})$ & $\{ \translt(\mve1) , \smalldots , \translt(\mven) \}$ & $\translt(H_1$ \twsml{and} $\textrm{\dots}$ \twsml{and} $H_n$\twsml{\lprl}$B.)$ & $\translt(H_1$\twsml{\lprl}$B.)\!$ , \smalldots , $\translt(H_n$\twsml{\lprl}$B.)$ \\
  $\translt(H_1$\twsml{\lprl}$H_2$\twsml{\lprl}$B.)$ & $\translt(H_1$\twsml{\lprl}$H_2$ \twsml{and} $B.)$ & $\translt(H$\twsml{\lprl} $B_1$ \twsml{or} , \smalldots , \twsml{or} $B_n.)$ & $\translt(H$\twsml{\lprl}$B_1.)\!$ , \smalldots , $\translt(H$\twsml{\lprl}$B_n.)$ \\
  \hline
\end{tabular}
\end{scriptsize}
\caption{Normalization of WSML logical expressions.}
\label{tab:normalization}
\end{table}
%
\\[2mm]
{\it Lloyd-Topor Transformation.} The transformation \translt is
applied as a mapping $\powset{\LE} \rightarrow \powset{\LE}$ to
flatten the complex WSML logical expressions, producing simple rules
according to the Lloyd-Topor transformations \cite{lloyd-topor}, as
shown in the lower part of Table~\ref{tab:normalization}. Again, the
meta variables $E_i,H_i,B_i$ range over WSML logical expressions,
while $H_i$ and $B_i$ match the form of valid rule head and body
expressions, respectively, according to \cite{wsml-spec}. After this
step, the resulting WSML expressions have the form of proper Datalog
rules with a single head and conjunctive (possibly negated) body
literals.
%
\\[2mm]
{\it Datalog Rule Generation.} In a final step, the transformation
\transdlog is applied as a mapping $\powset{\LE} \rightarrow \P$
from WSML logical expressions to the set of all Datalog programs,
yielding generic Datalog rules that represent the content of the
original WSML ontology. Rule-style language constructs, such as
rules, facts, constraints, conjunction and (default) negation, are
mapped to the respective Datalog elements. All remaining
WSML-specific language constructs, such as \wsml{subConceptOf} or
\wsml{ofType}, are replaced by special meta-level predicates for
which the semantics of the respective language construct is encoded
in meta-level axioms as described in Section~\ref{sec:meta}. The
right-hand part of Table \ref{tab:normalization} shows the mapping
from WSML logical expressions to Datalog including the meta-level
predicates \psco, \pmo, \phval, \pitype and \potype that represent
their respective WSML language constructs as can be seen from the
mapping. The meta variables $E,H,B$ range over WSML logical
expressions with a general, a head or a body form, while $C,I,a$
denote WSML concepts, instances and attributes. Variables $T$ can
either assume a concept or a datatype, and $V$ stands for either an
instance or a data value, accordingly.

The resulting Datalog rules are of the form $H \lprl B_1 \dlogand
\dots \dlogand B_n$, where $H$ and $B_i$ are literals for the head
and the body of the rule, respectively. Body literals can be
negated in the sense of negation-as-failure, which is denoted by
$\dlognot B_i$. As usual, rules with an empty body represent
facts, and rules with an empty head represent constraints. The
latter is denoted by the head being the empty clause symbol
$\square$.

Ultimately, we define the basic\footnote{Later on, the
transformation pipeline is further extended to support datatypes
and debugging.} transformation $\tau$ for converting a rule-based
WSML ontology into a Datalog program based on the single
transformation steps introduced before by $ \tau = \transdlog
\circ \translt \circ \transnorm \circ \transax$.
%\begin{displaymath}
%    \tau = \transdlog \circ \translt \circ \transnorm \circ \transax
%\end{displaymath}
As a mapping $\tau: \O \rightarrow \P$, this composition of the
single steps is applied to a WSML ontology $O \in \O$ to yield a
semantically equivalent Datalog program $\tau (O) = P \in \P$ when
interpreted with respect to the meta-level axioms discussed next.
