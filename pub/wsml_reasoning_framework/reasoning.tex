\subsection{WSML Reasoning by Datalog Queries}
%-- describe how to realise WSML satisfiability and entailment through Datalog querying \\
%-- -- characterize the KB (Datalog program) on which reasoning is performed with the different facts and rules  \\
%-- -- show how the WSML reasoning tasks are mapped to Datalog queries (KB sat., entailment and conjunctive query answering) \\

To perform reasoning over the original WSML ontology $O$ with an
underlying Datalog inference engine, a Datalog program $P_O =
\mlaxioms \cup \tau(O) $
%\compress
%\begin{displaymath}
%    P_O = \mlaxioms \cup \tau(O) \compress
%\end{displaymath}
is built up that consists of the meta-level axioms together with the
transformed ontology. The different WSML reasoning tasks are then
realized by performing Datalog queries on $P_O$. Posing a query
$Q(\vec{x})$ to a Datalog program $P \in \P$ is denoted by
$\,(P,\qury Q(\vec{x}))\,$ and yields the set of all tuples
$\vec{t}$ that instantiate the vector $\vec{x}$ of variables in the
query such that $Q(\vec{t})$ is satisfied in the well-founded model
of $P$. If $Q(\vec{x})$ contains no variables, in fact a boolean
query $Q$ is posed that instead evaluates either to $\{Q\}$ if $Q$
is satisfied in the well-founded model of $P$ or $\emptyset$
otherwise.\\[2mm]
%
{\it Ontology Consistency} -- The task of checking a WMSL ontology
for consistency is done by querying for the empty clause, as
expressed by the following equivalence: $ O \;
\textrm{\footnotesize{is satisfiable}} \; \Leftrightarrow \; (P_O ,
\qury \square) =
    \emptyset $
%\begin{displaymath}
%    O \; \textrm{\footnotesize{is satisfiable}} \; \Leftrightarrow \; (P_O , \qury \square) =
%    \emptyset \compress
%\end{displaymath}
. If the resulting set is empty then the empty clause could not be
derived from the program and the original ontology is satisfiable,
otherwise it is not.\\[2mm]
%
{\it Entailment} -- The reasoning task of ground entailment by a
WSML ontology is done by using queries that contain no variables, as
expressed in the following equivalence: $O \models \phi_g \;
\Leftrightarrow \; (P_O, \qury \tau'(\phi_g)) ) \not= \emptyset$.
%\compress
%\begin{displaymath}
%    O \models \phi_g \; \Leftrightarrow \; (P_O, \qury
%    \tau'(\phi_g)) ) \not= \emptyset \compress
%\end{displaymath}
The WSML ground fact $\phi_g \in \LE$ is transformed to Datalog
with a transformation $\tau' = \transdlog \circ \translt \circ
\transnorm$, similar to the one that is applied to the ontology,
and is evaluated together with the Datalog program $P_O$. If the
resulting set is non-empty then $\phi_g$ is entailed by the
original ontology, otherwise it is not.\\[2mm]
%
{\it Retrieval} -- Similarly, instance retrieval can be performed by
posing a WSML query $Q(\vec{x})$ with free variables $\vec{x}$ to
the Datalog program $P_O$, which yields the following set:
$\{\vec{o}  \, | \,  O \models Q(\vec{o})\} \; = \; (P_O, \qury
\tau'(Q(\vec{x})))$.
%\compress
%\begin{displaymath}
%   \{\vec{o} : O \models Q(\vec{o})\} \; = \; (P_O, \qury
%   \tau'(Q(\vec{x}))) \compress
%\end{displaymath}
The query $Q(\vec{x})$ is transformed to Datalog by $\tau'$ and
evaluated together with the program $P_O$. The resulting set
contains all object tuples $\vec{o}$ for which an instantiation of
the query expression is entailed by the original ontology, while
the objects in $\vec{o}$ can be identifiable WSML entities or data
values. For
example, the query \thinspace $Q($\wsmlname{?x}$)$ = %\phantom{mmmmm}
\wsmlname{?x} \wsml{memberOf} \wsmlname{BroadbandBundle}
\thinspace posed to the ontology in Listing
\ref{lst:wsml-ontology-example} yields the set $\{
(\textrm{\wsmlname{MyBundle}}) \}$ that contains one unary tuple
with the instance \wsmlname{MyBundle}, which can be inferred to be
a broadband bundle due to its high network bandwidth.
