\subsection{WSML Reasoning by Datalog Queries}
%-- describe how to realise WSML satisfiability and entailment through datalog querying \\
%-- -- characterize the KB (datalog program) on which reasoning is performed with the different facts and rules  \\
%-- -- show how the WSML reasoning tasks are mapped to datalog queries (KB sat., entailment and conjunctive query answering) \\

To perform reasoning over the original WSML ontology $O$ with an
underlying datalog inference engine, a datalog program
\begin{displaymath}
    P_O = \mlaxioms \cup \tau(O)
\end{displaymath}
is built up that consists of the meta-level axioms together with
the transformed ontology. The different WSML reasoning tasks are
then realized by performing Datalog queries on $P_O$. Posing a
query $Q(\vec{x})$ to a Datalog program $P \in \P$ is denoted by
$$(P,\qury Q(\vec{x}))$$ and yields a set of tuples that instantiate
the vector $\vec{x}$ of variables in the query.

\paragraph{Ontology Consistency} -- The task of checking a WMSL
ontology for consistency is done by querying for the empty clause,
as expressed by the following equivalence.
\begin{displaymath}
    O \; \textrm{\footnotesize{is satisfiable}} \; \Leftrightarrow \; (P_O , \qury \square) =
    \emptyset
\end{displaymath}
If the resulting set is empty then the empty clause could not be
derived from the program and the original ontology is satisfiable,
otherwise it is not.

\paragraph{Entailment} -- The reasoning task of entailment of
ground facts by a WSML ontology can be done by using queries that
contain no variables, as expressed in the following equivalence.
\begin{displaymath}
    O \models \phi \; \Leftrightarrow \; (P_O, \qury
    \tau(\phi)) \not= \emptyset
\end{displaymath}
The WSML ground fact $\phi \in \LE$ is transformed to Datalog with
a transformation $\transdlog \circ \translt \circ \transnorm$,
similar to the one that apply to the ontology, and is evaluated
together with the Datalog program $P_O$. If the resulting set is
non-empty then $\phi$ is entailed by the original ontology,
otherwise it is not.

\paragraph{Retrieval} -- Similarly, instance retrieval can be
performed by posing queries that contain variables to the Datalog
program $P_O$, as expressed in the following equivalence.
\begin{displaymath}
   % \{\vec{x} : O \models Q(\vec{x})\} \; \Leftrightarrow \; (P_O, \qury \tau(Q(\vec{x})))
   retrieve_O(Q) \; = \; (P_O, \qury \tau(Q(\vec{x})))
\end{displaymath}
The query $Q(\vec{x})$, formulated as a WSML logical expression
with free variables $\vec{x}$, is transformed to Datalog and
evaluated together with the program $P_O$. The resulting set
contains all tuples $\vec{x}$ for which an instantiation of the
query expression is entailed by the original ontology.

To give an example, the query\\
\phantom{mmmmm} \syn{?x} \synkw{memberOf} \syn{BroadbandBundle}\\
posed to the ontology from Section \ref{sec:wsml} yields the set
$\{ (\textit{MyBundle}) \}$ that contains one unary tuple with the
instance \textit{MyBundle}, which can be inferred to be a
broadband bundle due to its high network bandwidth.

%\begin{small}
%\begin{tabular}{|l|l|}
%  \hline
%  $O$ is satisfiable & $(P_O, \qury \dlognot \square) \rightarrow \top$ \\
%  $O \models \phi(\vec{C})$ & $(P_O, \qury \phi(\vec{C})) \rightarrow \top$ \\
%  $\{\vec{X} : O \models Q(\vec{X})\}$ & $\{\vec{X} : (P_O, \qury Q(\vec{X})) \rightarrow \top\}$ \\
% \hline
%\end{tabular}
%\end{small}
%
%( $\phi_g$ : ground fact ; $\vec{X}$ : variable binding )
