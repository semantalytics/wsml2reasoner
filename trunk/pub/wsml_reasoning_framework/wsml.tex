\newcommand{\concept}[1]{\wsmlname{#1}}
\newcommand{\attribute}[1]{\wsmlname{#1}}
\newcommand{\instance}[1]{\wsmlname{#1}}

\section{The WSML Language}\label{sec:wsml}
The Web Service Modeling Language (WSML) \cite{wsml} is a language
for the specification of various aspects of Semantic Web Services
(SWS), such as what functionality is provided by a SWS or how to
interact with the SWS. It provides a formal language for the Web
Service Modeling Ontology
(WSMO\footnote{\url{http://www.wsmo.org}})~\cite{roman05:_web_servic_model_ontol}
and is based on well-known logic-based knowledge representation
(KR) formalisms, namely Description
Logics~\cite{Baader+CalvaneseETAL-DescLogiHand:03} and Logic
Programming~\cite{lloyd-FoundationsOfLP}. In fact, WSML is a
family of representation languages that comes in several variants
with different expressiveness. Besides various SWS-specific
language constructs, such as ``goal", ``interface",
``choreography" or ``capability", WSML particularly provides means
to formulate the domain ontologies in terms of which SWSs are
semantically annotated. Since here we are interested in reasoning
with such semantic annotation with respect to the underlying
ontology formalism, we focus on the ontology-related part of WSML.
Furthermore, we use the human-readable syntax of the language in
our presentation, while WSML also specifies XML and RDF
serialisations to be compatible with existing web standards.

\subsection{Language Constructs}
WSML makes a clear distinction between the modeling of different
conceptual elements on the one hand and the specification of complex
axiomatic information on the other. To this end, the WSML syntax is
split into two parts: the \emph{conceptual syntax},  and
\emph{logical expression syntax}, while elements from both can be
combined in a WSML document. We illustrate the interplay of
conceptual modelling with logical expressions in WSML by means of an
example given in Listing~\ref{lst:wsml-ontology-example}, which
specifies an ontology in the domain of telecommunications taken from
a project case study. For a complete account of all WSML syntax
elements, we refer to \cite{wsml-spec}.
\begin{lstlisting}[label=lst:wsml-ontology-example,style=wsml, caption=WSML Example Ontology]
concept Product
    hasProvider inverseOf(Provider#provides) impliesType Provider
concept ITBundle subConceptOf Product
    hasNetwork ofType (0 1) NetworkConnection
    hasOnlineService ofType (0 1) OnlineService
    hasProvider impliesType TelecomProvider
concept NetworkConnection subConceptOf BundlePart
    providesBandwidth ofType (0 1) _integer
concept DialupConnection subConceptOf NetworkConnection
concept DSLConnection subConceptOf NetworkConnection
axiom DialupConnection_DSLConnection_Disjoint definedBy
 !- ?x memberOf DialupConnection and ?x memberOf DSLConnection.
concept OnlineService subConceptOf BundlePart concept
SharePriceFeed subConceptOf OnlineService axiom
SharePriceFeed_requires_bandwidth definedBy
 !- ?b memberOf ITBundle and ?b[hasOnlineService hasValue ?o]
    and ?o memberOf SharePriceFeed and
    ?b[hasNetwork hasValue ?n] and
    ?n[providesBandwidth hasValue ?x] and ?x < 512.
concept BroadbandBundle subConceptOf ITBundle
    hasNetwork ofType (1 1) DSLConnection
axiom BroadbandBundle_sufficient_condition definedBy
    ?b memberOf BroadbandBundle :- ?b memberOf ITBundle
    and ?b[hasNetwork hasValue ?n] and ?n memberOf DSLConnection.
instance BritishTelekom memberOf TelecomProvider.
instance UbiqBankShareInfo memberOf SharePriceFeed.
instance MyBundle memberOf ITBundle
    hasNetwork hasValue ArcorDSL
    hasOnlineService hasValue UbiqBankShareInfo
    hasProvider BritishTelekom.
instance MSNDialup memberOf DialupConnection
    providesBandwidth hasValue 10.
instance ArcorDSL memberOf DSLConnection
    providesBandwidth hasValue 1024.
\end{lstlisting}

{\bfseries Conceptual Modelling.} The WSML conceptual syntax for
ontologies essentially allows for the modeling of concepts,
instances and relations.

In ontologies, concepts form the basic elements for describing the
terminology of the domain of discourse by means of classes of
objects. In the telecommunications domain, a concept like
\concept{NetworkConnection} stands for the class of all network
connections. Concepts can be put in a subsumption hierarchy by
means of the \wsml{subConceptOf}-construct. For example,
\concept{NetworkConnection} is a subconcept of
\concept{BundlePart}, meaning that any network connection is part
of some IT product bundle, and has itself the subconcepts
\concept{DialupConnection} and \concept{DSLConnection}, as can be
seen from Listing~\ref{lst:wsml-ontology-example}.

Attributes, i.e.\ binary relations, are used to relate concepts in a
customary way, while they can point to other concepts or datatypes.
In our example, \concept{NetworkConnection} has a datatype attribute
\attribute{providesBandWidth}, whereas concept \concept{ITBundle}
has attributes like \attribute{hasNetwork} or
\attribute{hasOnlineService} that point to concepts for the single
parts which make up the bundle. Attribute definitions can either be
\emph{constraining} (using \wsml{ofType}) or \emph{inferring} (using
\wsml{impliesType})\footnote{The distinction
  between inferring and constraining attribute definitions is
  explained in more detail in \cite[Section
  2]{debr-etal-2005c}.}. Constraining attribute definitions
define a type constraint on the values for an attribute, similar
to integrity constraints in databases; inferring attribute
definitions allow that the type of the values for the attribute is
inferred from the attribute definition, similar to range
restrictions on properties in
RDFS~\cite{Brickley+Guha-VocaDescLang:03} and
OWL~\cite{Dean+Schreiber-OntoLangRefe:04}. Furthermore, an
attribute can be marked as \wsml{transitive}, \wsml{symmetric}, or
\wsml{reflexive}, and can be constrained by a minimum and a
maximum cardinality (using \begin{footnotesize}\textsf{($n_{min}$
$n_{max}$)}\end{footnotesize}), as can be seen from
Litsing~\ref{lst:wsml-ontology-example}. Similar constructs are
available to define n-ary relations in ontologies.

Instances represent concrete objects a the domain, such as
\instance{MSNDialup} as a particular dial-up connection in the
telecommunications domain. By means of the
\wsml{memberOf}-construct, instances are associated with concepts,
and using \wsml{hasValue} they are linked to other instances or
data values, as can also be seen in
Listing~\ref{lst:wsml-ontology-example}. Notice, that WSML
supports metamodelling and allows an entity to be both a concept
and an
instance.\\[2mm]
{\bfseries Logical Expressions.} By means of the
\wsml{axiom}-construct, arbitrarily complex logical expressions can
be included in a WSML ontology, interfering with the conceptual
definitions. In our example, the axiom named
\wsmlname{BroadbandBundle\_sufficient\_condition} specifies that any
IT bundle that has a DSL network connection is concluded to be a
broadband bundle.

The general logical expression syntax for WSML has a first-order
logic style, in the sense that it has constants, function symbols,
variables, predicates and the usual logical connectives.
Additionally, WSML provides extensions based on F-Logic
\cite{Kifer+LausenETAL-LogiFounObjeFram:95} as well as logic
programming rules and database-style integrity constraints.

Besides standard first-order atoms, WSML provides so-called
\emph{molecules}, inspired by F-Logic, that can be used to capture
information about concepts, instances, attributes and attribute
values. A molecule of the form $I$ \wsml{memberOf} $C$ denotes the
membership of an instance $I$ in a concept $C$, while a molecule
$C_1$ \wsml{subConceptOf} $C_2$ denotes the subconcept relationship
between concepts $C_1$ and $C_2$. Further molecules have the form
$I$[$A$ \wsml{hasValue} $V$] to denote attribute values of objects,
$C$[$A$ \wsml{ofType} $T$] to denote a type-constraining attribute
signature, or $C$[$A$ \wsml{impliesType} $T$] to denote an inferring
attribute signature. Some of these molecule forms appear in
Listing~\ref{lst:wsml-ontology-example}, e.g. in axiom
\wsmlname{BroadbandBundle\_sufficient\_condition}.

WSML has the usual first-order connectives: the unary (classical)
negation operator \wsml{neg}, and the binary operators for
conjunction \wsml{and}, disjunction \wsml{or}, right implication
\wsml{implies}, left implication \wsml{impliedBy}, and
bi-implication \wsml{equivalent}. Variables, preceeded by the
\wsmlname{?}-symbol may be universally quantified using
\wsml{forall} or existentially quantified using \wsml{exists}. Apart
from first-Order constructs, WSML supports logic programming rules
of the form $H :\!- B$ with the typical restrictions on the head and
body expressions $H$ and $B$ (see~\cite{wsml-spec}), allowing the
symbol \wsml{naf} for negation-as-failure on atoms in $B$. A
constraint is a special kind of rule with an empty head expression.
While the aforementioned axiom is expressed by a rule, the axiom
named \wsmlname{DialupConnection\_DSLConnection\_Disjoint} comes in
form of a constraint, stating that no instance is allowed to be
member of both the concepts \concept{DialupConnection} and
\concept{DSLConnection} at the same time.\\[2mm]
{\bfseries Language Variants.} WSML comes in different variants that
map to semantically different target formalisms. Therefore, each
variant also defines some restrictions on the use of syntactical
constructs: \noindent {\sl \bfseries WSML-Core} allows only
first-order formulae which conform to DLP~\cite{dlp} as the least
common denominator of the description logics and logic programming
paradigms, by which its semantics is defined. It allows for most of
conceptual modelling but is rather restricted in the use of logical
expressions. {\sl \bfseries WSML-DL} allows first-order formulae
which can be translated to the description logic
$\mathcal{SHIQ}(\mathbf{D)}$, that defines its semantics. Thus,
WSML-DL is very similar to
OWL~\cite{Dean+Schreiber-OntoLangRefe:04}. {\sl \bfseries
WSML-Flight} extends WSML-Core by allowing variables in place of
instance, concept and attribute identifiers and by allowing
relations of arbitrary arity. In fact, any such formula is allowed
in the head of a WSML-Flight rule. The body of a WSML-Flight rule
allows conjunction, disjunction and default negation. WSML-Flight is
based on the well-founded
semantics~\cite{Gelder+RossETAL-WellSemaGeneLogi:91} and
additionally allows meta-modeling. {\sl \bfseries WSML-Rule} extends
WSML-Flight by function symbols and unsafe rules, i.e.\ variables
occurring in the head or in a negative body literal but not in a
positive body literal. {\sl \bfseries WSML-Full} does not restrict
the use of syntax and allows the full expressiveness of all other
WSML variants under a first-order umbrella with nonmonotonic
extensions.

In the following, we refer to the WSML-Core, WSML-Flight and
WSML-Rule variants jointly as \emph{rule-based WSML} and focus on
reasoning in these variants.

\subsection{Reasoning in Rule-Based WSML}
Various reasoning tasks, such as consistency checking or
entailment of implicit knowledge, are considered useful in
Semantic Web and SWS applications. Here, we sketch the typical
reasoning tasks for rule-based formalisms, and thus for rule-based
WSML.

Let $O$ denote a rule-based WSML ontology and $\pi_{c-free}(O)$
denote the constraint-free projection of $O$, i.e. the ontology
which is obtained from $O$ by removing all constraining
description elements, such as attribute type constraints,
cardinality constraints, integrity constraints etc. (1) {\bf
Consistency checking} means to verify whether $O$ is satisfiable,
i.e. if $\pi_{c-free}(O)$ has a model in which no constraint in
$O$ is violated. (2) {\bf Ground Entailment} means, given some
variable-free formula $\phi_g$, to check if $\phi_g$ is satisfied
in well-founded model of $\pi_{c-free}(O)$ in which no constraint
in $O$ is violated. We denote this by $O \models \phi_g$. (3) {\bf
Instance Retrieval} means, given an ontology $O$ and some formula
$Q(\vec{x})$ with free variables $\vec{x} = (x_1,\ldots,x_n)$, to
find all suitable terms $\vec{t} = (t_1,\ldots,t_n)$ constructed
from symbols in $O$ only, such that $O \models Q(\vec{t})$.
