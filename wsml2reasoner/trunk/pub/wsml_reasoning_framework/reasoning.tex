\subsection{WSML Reasoning by Datalog Queries}
%-- describe how to realise WSML satisfiability and entailment through datalog querying \\
%-- -- characterize the KB (datalog program) on which reasoning is performed with the different facts and rules  \\
%-- -- show how the WSML reasoning tasks are mapped to datalog queries (KB sat., entailment and conjunctive query answering) \\

To perform reasoning over the original WSML ontology $O$ with an
underlying datalog inference engine, a datalog program
\begin{displaymath}
    P_O = \mlaxioms \cup \tau(O)
\end{displaymath}
is build up that consists of the meta-level axioms together with
the transformed ontology. The different WSML reasoning tasks are
then realized by performing datalog queries on $P_O$. A Query
$Q(\vec{x})$ posed to a Datalog program $P$ is denoted by
$$(P,\qury Q(\vec{x}))$$ and yields a set of tuples that instantiate
the variable vector $\vec{x}$.

\subparagraph{Ontology Consistency} -- The task of checking a WMSL
ontology for consistency is done by querying for the empty clause,
as expressed by the following equivalence.
\begin{displaymath}
    O \; \textrm{\footnotesize{is satisfiable}} \; \Leftrightarrow \; (P_O , \qury \square) =
    \emptyset
\end{displaymath}
If the resulting set is empty then the empty clause could not be
derived from the program and the original ontology is satisfiable.

\subparagraph{Entailment} -- The reasoning task of entailment of
ground facts by a WSML ontology can be done by using queries that
contain no variables, as expressed in the following equivalence.
\begin{displaymath}
    O \models \phi \; \Leftrightarrow \; (P_O, \qury
    \tau(\phi)) \not= \emptyset
\end{displaymath}
The WSML ground fact $\phi$ is transformed to Datalog and
evaluated together with the Datalog program $P_O$. If the
resulting set is non-empty then $\phi$ is entailed by the original
ontology.

\subparagraph{Retrieval} -- Similarly, instance retrieval can be
performed by posing queries that contain variables to the Datalog
program $P_O$, as expressed in the following equivalence.
\begin{displaymath}
    \{\vec{x} : O \models Q(\vec{x})\} \; \Leftrightarrow \; (P_O, \qury \tau(Q(\vec{x})))
\end{displaymath}
The query $Q(\vec{x})$, formulated as a WSML logical expression
with free variables $\vec{x}$, is transformed to Datalog and
evaluated together with the Datalog program $P_O$. The resulting
set contains all tuples $\vec{x}$ for which the query expression
is entailed by the original ontology.

%\begin{small}
%\begin{tabular}{|l|l|}
%  \hline
%  $O$ is satisfiable & $(P_O, \qury \dlognot \square) \rightarrow \top$ \\
%  $O \models \phi(\vec{C})$ & $(P_O, \qury \phi(\vec{C})) \rightarrow \top$ \\
%  $\{\vec{X} : O \models Q(\vec{X})\}$ & $\{\vec{X} : (P_O, \qury Q(\vec{X})) \rightarrow \top\}$ \\
% \hline
%\end{tabular}
%\end{small}
%
%( $\phi_g$ : ground fact ; $\vec{X}$ : variable binding )
