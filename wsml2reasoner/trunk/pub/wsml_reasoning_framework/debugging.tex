\def\transdebug{\transtxt{debug}}
\def\pvotype{\predicate{\predsubtxt{v\_otype}}}
\def\pvmincard{\predicate{\predsubtxt{v\_mincard}}}
\def\pvmaxcard{\predicate{\predsubtxt{v\_maxcard}}}
\def\pvuser{\predicate{\predsubtxt{v\_user}}}
\def\axiomid{\ensuremath{Ax_{I\!D}}\xspace}
\def\debugaxioms{\ensuremath{P_{\smtxtit{debug}}}\xspace}

\section{Debugging Support\label{sec:debugging}}
%-- briefly motivate debugging for the ontology engineering process \\
%-- -- during the process of ontology engineering, a modeller easily produces erroneous contradictory information, which he needs to get aware of \\
%-- -- the source for contradictory information in rule-based WSML are primarily constraints \\
%-- -- with plain datalog mechanisms, modellers only get aware of whether some constraint is violated, i.e. whether the empty clause is derived from $P_O$ indicating that the original ontology $O$ contains erroneous modelling; the modeller is not informed about the type of problem and why the problem occurs \\
%-- -- our framework offers debugging features that allow for identifying violated constraints and involved WSML entities, which is achieved by replacing constraints by rules with special additional meta-level predicates in their head \\

During the process of ontology development, an ontology engineer
can easily construct a situation of erroneous modelling with
contradictory information. In order to produce consistent
ontologies, engineers need to be pointed out to such defects with
some detail about the involved ontological elements that cause the
problem.

In rule-based WSML, the source for erroneous modelling by
contradictory information are primarily constraints together with
a violating situation of concrete instances related via
attributes. The plain datalog mechanisms employed in the reasoning
framework according to Section \ref{sec:mapping} only allow for
checking whether some constraint is violated, i.e.\ whether the
empty clause is derived from $P_O$ indicating that the original
ontology $O$ contains erroneous modelling; more detailed
information about the problematic situation is not reported per
se. In our framework, we support debugging features that allow for
identifying violated constraints together with the ontological
entities involved, which we achieve by replacing constraints with
appropriate rules that build on additional meta-level predicates.

\subsection{Identifying Constraint Violations}
%-- describe the kind of debugging features the framework supports and what they allow for \\
%-- -- in a WMSL ontology, constraints can be violated by the instance situation, making the ontology inconsistent \\
%-- -- two things a of interest to the modeller when a constraint is violated: a) the type of constraint and the entities involved \\
%-- -- $<$ give an example of a violated constraint to illustrate what information is relevant for the modeller in such a situation $>$ \\
%-- -- in the different situations of violation of different types of constraints, different information is relevant for the modeller to repair the erroneous situation  \\
%-- -- $<$ list different types of violations $>$ \\

An ontology in rule-based WSML is inconsistent when constraints
are violated by the particular situation of its instances and
their relation via attributes. In such a violating situation, two
things are of interest to the ontology engineer: a) the type of
constraint that is violated and b) the entities, i.e.\ concepts,
attributes, instances, etc., that are involved in the violation.

\def\bla{\textbf{{\sf bla}}\xspace}
To give an example, consider the WSML ontology in Section
\ref{sec:wsml}. There, the attribute \bla of the concept \bla is
constrained to instances of type \bla. If we would add \bla, this
constraint would be violated because \bla cannot be derived to be
an instance of the concept \bla. For an ontology engineer who
needs to repair this erroneous modelling, it is important to know
the entities that cause the violation, which in this case are the
attribute \bla together with the range concept \bla and the
non-conforming instance \bla. In different situations of violation
of the various kinds of constraints, the information important for
the ontology engineer to know is also different, as listed in the
following.

\subparagraph{Attribute Type Violation} -- An attribute type
constraint of the form $C[a$ \wsml{ofType} $T]$ is violated in any
situation where an instance of the concept $C$ has value $V$ for
the attribute $a$ but where $V$ cannot be derived to belong to the
extension of the type $T$. Here, $T$ can be either a concept or a
datatype, while $V$ is then an instance or a data value,
accordingly. In such a situation, an ontology engineer is
particularly interested in the attribute value $V$ that caused the
constraint violation, together with the attribute $a$ and the
expected type $T$ which the value $V$ failed to adhere to.

\subparagraph{Minimum Cardinality Violation} -- A minimum
cardinality constraint, contained in a concept definition of the
form \wsml{concept} $C$ $a (n *)$, is violated in any situation
where the number of distinguished values of the attribute $a$ for
some instance $I$ of the concept $C$ is less than the specified
cardinality $n$. In such a situation, an ontology engineer is
particularly interested in the instance $I$ that failed to have a
sufficient number of attribute values, together with the actual
attribute $a$. (Information about how many values were missing can
be learned by querying the ontology separately.)

\subparagraph{Maximum Cardinality Violation} -- A maximum
cardinality constraint, contained in a concept definition of the
form \wsml{concept} $C$ $a (0 n)$, is violated in any situation
where the number of distinguished values of the attribute $a$ for
some instance $I$ of the concept $C$ exceeds the specified
cardinality $n$. Again, here an ontology engineer is particularly
interested in the instance $I$ for which the number of attribute
values was exceeded, together with the actual attribute $a$.

\subparagraph{User-Defined Constraint Violation} -- A user-defined
constraint, contained in an axiom definition of the form
\wsml{axiom} \axiomid \wsml{definedBy} $B.$, is violated in any
situation where the body $B$ of the constraint evaluates to true
for any instantiation of its free variables. In such a situation,
the information which helps an ontology engineer to repair an
erroneous situation is specific to the user-defined constraint
given by $B$ and cannot be reported in a generic manner. However,
a generic framework can at least identify the violated constraint
by pointing out the name \axiomid of the axiom.

\subsection{Debugging by Meta-Level Reasoning}
%-- describe how these debugging features are realised via additional meta-level predicates an additional fixed set of rules \\
%-- -- in our framework we realise the debugging features for identifying constraint violations together with involved entities by replacing constraints with rules \\
%-- -- these rules have additional debugging-specific meta-level predicates in their heads which are instantiated when a constraint body evaluates to true; this way the debugging information is derived by datalog rules and can be queried for \\
%-- -- the replacements of constraints is included in the transformation pipeline $\tau$ as an additional step \\
%\begin{displaymath}
%    \tau = \transdlog \circ \translt \circ \transnorm \circ \transdebug \circ \transax
%\end{displaymath}
%-- -- the additional transformation step $\transdebug$ is applied after conceptual syntax has been resolved, replacing constraints on the level of WSML logical expressions \\
%-- -- the detailed constrained replacement performed by \transdebug can be seen from Table \ref{tab:debugging} \\
%-- -- the body variables are supposed to match the appropriate form of constraint body; notice: the semantics of ofType is encoded in the meta-level axioms \mlaxioms, so ofType-constraints can't be as easily replaced but have to be generated by \transdebug \\
%-- -- to maintain the constraint-semantics, some additional debugging-specific meta-level axioms, denoted by \debugaxioms, have to be included, which are shown in Table \ref{tab:debugging-axioms} \\
%-- -- thus, the datalog program used for reasoning with the original WSML ontology turns to: \\
%\begin{displaymath}
%    P_O = \mlaxioms \cup \debugaxioms \cup \tau(O)
%\end{displaymath}
%-- -- then one can ask for occurrences of the different kinds of constraint violation by e.g. \\
%\begin{displaymath}
%    \{(a,T,I,V) : (P_O , ?\pvotype(a,T,I,V)) \rightarrow \top \}
%\end{displaymath}
%-- -- which asks for all occurrences of type violations by means of datalog querying mechanisms; if this set is empty then there is no problem concerning types \\

In our framework, we realize the debugging features for pointing
out constraint violations by replacing constraints with a special
kind of rules. Instead of deriving the empty clause, as
constraints do, these rule derive information about occurrences of
constraint violations by instantiating debugging-specific
meta-level predicates with the entities involved in a violation.
In this way, information about constraint violations can be
queried for by means of datalog inferencing.

The replacement of constraints for debugging is included in the
transformation pipeline
\begin{displaymath}
    \tau = \transdlog \circ \translt \circ \transnorm \circ \transdebug \circ \transax
\end{displaymath}
where the additional transformation step \transdebug is applied
after the WSML conceptual syntax has been resolved, replacing
constraints on the level of WSML logical expressions. Table
\ref{tab:debugging} shows the detailed replacements performed by
\transdebug for the different kinds of constraints.

The two types of cardinality constraints with bodies $B_{mincard}$
and $B_{maxcard}$ are turned into rules by keeping the respective
body and including a head that instantiates one of the predicates
\pvmincard and \pvmaxcard to indicate the respective cardinality
violation. The meta variables $B_{<X>}$ are supposed to
syntactically match the respective bodies for cardinality
constraints produced by \transax (Table
\ref{tab:conceptual-logical}), in which the arguments for the
meta-level predicates, namely the involved attribute $a$ and
instance $I$, occur.

Similarly, a user-defined constraint is turned into a rule by
keeping the predefined body $B_{user}$ and including a head that
instantiates the predicate \pvuser to indicate a user-defined
violation. Since the body $B_{user}$ can be an arbitrarily complex
logical expression, the only argument of this predicate is the
identifier \axiomid of the axiom that contains the  specification
of the user-defined constraint.

Constraints on attribute types are handled in a different way
because these constraints are not expanded during the
transformation \transax ; they are rather represented by WSML
\wsml{ofType}-molecules for which the semantics is encoded in the
meta-level axioms \mlaxioms. In order to avoid the modification of
\mlaxioms in the reasoning framework, such molecules are expanded
by \transdebug, as shown in Table
\ref{tab:debugging}.\footnote{After this expansion of
\wsml{ofType} molecules, the respective axiom (4) in \mlaxioms for
realising the semantics of attribute type constraints does not
apply anymore.}

\medskip

To maintain the constraining semantics of the replaced
constraints, an additional set of meta-level axioms $\debugaxioms
\in \mathcal{P}$ is included for reasoning, which derive the empty
clause for any occurrence of a constraint violation, as shown in
Table \ref{tab:debugging-axioms}. Including the debugging
features, the datalog program for reasoning about the original
ontology then turns to
\begin{displaymath}
    P_O = \mlaxioms \cup \debugaxioms \cup \tau(O) \;\;\;.
\end{displaymath}
Occurrences of constraint violations can be recognized by querying
$P_O$ for instantiations of the various debugging-specific
meta-level predicates \pvotype, \pvmincard, \pvmaxcard and
\pvuser. For example, the set
\begin{displaymath}
    \{(a,T,I,V) : (P_O , \qury \pvotype(a,T,I,V)) \rightarrow \top \}
\end{displaymath}
contains tuples for all occurrences of attribute type violations
in $P_O$, identifying the respective attribute $a$, expected type
$T$, involved instance $I$ and violating value $V$ for each. This
set is empty if there are no attribute types violated.


\begin{table}[tb]\label{tab:debugging}\centering
\begin{footnotesize}
\begin{tabular}{|l|l|}
  \hline
  \rule{0cm}{3.2mm} {\normalsize \emph{Constraint}} & {\normalsize \emph{Rule}} \\
  \hline
  $\transdebug($\wsml{\cstr}$B_{mincard}.)$ & $\pvmincard(a,I)$\wsml{\lprl}$B_{mincard}.$ \\
  $\transdebug($\wsml{\cstr}$B_{maxcard}.)$ & $\pvmaxcard(a,I)$\wsml{\lprl}$B_{maxcard}.$ \\
  $\transdebug($\wsml{\cstr}$B_{user}.)$ & $\pvuser(\axiomid)$\wsml{\lprl}$B_{user}.$ \\
  $\transdebug(C[a$ \wsml{ofType} $T.)$ & $\pvotype(a,T,I,V)$\wsml{\lprl} \\
  & $\;C[a$ \wsml{ofType} $T]$ \wsml{and} $I$ \wsml{memberOf} $C$ \\
  & $\;I[a$ \wsml{hasValue} $V]$ \wsml{and naf} $V$\wsml{memberOf} $T.$ \\
  \hline
\end{tabular}
\end{footnotesize}
\caption{Replacing constraints by rules for debugging.}
\end{table}

\begin{table}[tb]\label{tab:debugging-axioms}\centering
\begin{small}
\begin{tabular}{|ll|}
  \hline
  \multicolumn{2}{|l|}{\rule{0cm}{3.2mm}{\normalsize \emph{Debugging Meta-Level Axioms}}} \\
  \hline
  (1) & $\dlogcstr \pvotype(a,T,I,V)$ \\
  (2) & $\dlogcstr \pvmincard(a,I)$ \\
  (3) & $\dlogcstr \pvmaxcard(a,I)$ \\
  (4) & $\dlogcstr \pvuser(\axiomid)$ \\
 \hline
\end{tabular}
\end{small} \caption{Meta-level axioms for WSML semantics in
datalog.}
\end{table}
