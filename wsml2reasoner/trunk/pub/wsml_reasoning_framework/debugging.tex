\def\transdebug{\transtxt{debug}}
\def\pvotype{\predicate{\predsubtxt{v\_otype}}}
\def\pvmincard{\predicate{\predsubtxt{v\_mincard}}}
\def\pvmaxcard{\predicate{\predsubtxt{v\_maxcard}}}
\def\pvuser{\predicate{\predsubtxt{v\_user}}}
\def\axiomid{\ensuremath{Ax_{I\!D}}}
\def\debugaxioms{\ensuremath{P_{\smtxtit{debug}}}\xspace}

\section{Debugging Support\label{sec:debugging}}
-- briefly motivate debugging for the ontology engineering process \\
-- -- during the process of ontology engineering, a modeller easily produces erroneous contradictory information, which he needs to get aware of \\
-- -- the source for contradictory information in rule-based WSML are primarily constraints \\
-- -- with plain datalog mechanisms, modellers only get aware of whether some constraint is violated, i.e. whether the empty clause is derived from $P_O$ indicating that the original ontology $O$ contains erroneous modelling; the modeller is not informed about the type of problem and why the problem occurs \\
-- -- our framework offers debugging features that allow for identifying violated constraints and involved WSML entities, which is achieved by replacing constraints by rules with special additional meta-level predicates in their head \\

\subsection{Identifying Constraint Violations}
-- describe the kind of debugging features the framework supports and what they allow for \\
-- -- in a WMSL ontology, constraints can be violated by the instance situation, making the ontology inconsistent \\
-- -- two things a of interest to the modeller when a constraint is violated: a) the type of constraint and the entities involved \\
-- -- $<$ give an example of a violated constraint to illustrate what information is relevant for the modeller in such a situation $>$ \\
-- -- in the different situations of violation of different types of constraints, different information is relevant for the modeller to repair the erroneous situation  \\
-- -- $<$ list different types of violations $>$ \\

    \subparagraph{Attribute Type Violation} -- An attribute type
    constraint of the form $C[a$ \wsml{ofType} $T]$ is violated in
    any situation where an instance of the concept $C$ has value
    $V$ for the attribute $a$ but where $V$ cannot be derived to
    belong to the extension of the type $T$. Here, $T$ can be
    either a concept or a datatype, while $V$ is then an instance
    or a data value, accordingly. In such a situation, an ontology
    engineer is particularly interested in the attribute value $V$
    that caused the constraint violation, together with the
    attribute $a$ and the expected type $T$ which the value $V$
    failed to adhere to.

    \subparagraph{Minimum Cardinality Violation} -- A minimum
    cardinality constraint, contained in a concept definition of
    the form \wsml{concept} $C$ $a (n *)$, is violated in any
    situation where the number of distinguished values of the
    attribute $a$ for some instance $I$ of the concept $C$ is less
    than the specified cardinality $n$. In such a situation, an
    ontology engineer is particularly interested in the instance
    $I$ that failed to have a sufficient number of attribute
    values, together with the actual attribute $a$. (Information about
    how many values were missing can be learned by querying the
    ontology separately.)

    \subparagraph{Maximum Cardinality Violation} -- A maximum
    cardinality constraint, contained in a concept definition of
    the form \wsml{concept} $C$ $a (0 n)$, is violated in any
    situation where the number of distinguished values of the
    attribute $a$ for some instance $I$ of the concept $C$ exceeds
    the specified cardinality $n$. Again, here an ontology
    engineer is particulary interested in the instance $I$ for which
    the number of attribute values has been exceeded, together the
    actual attribute $a$.

    \subparagraph{User-Defined Constraint Violation} -- A
    user-defined constraint, contained in an axiom definition of
    the form \wsml{axiom} \axiomid \wsml{definedBy} $B.$, is
    violated in any situation where the body $B$ of the constraint
    evaluates to true for any instantiation of its free variables.
    In such a situation, the information which helps an ontology
    engineer to repair an erroneous situation is specific to the user-defined
    constraint given by $B$ and cannot be specified in a generic
    manner. However, a generic framework can at least identify the
    violated constraint by pointing out the name \axiomid of the axiom.


\subsection{Debugging through Meta-Level Reasoning}
-- describe how these debugging features are realised via additional meta-level predicates an additional fixed set of rules \\
-- -- in our framework we realise the debugging features for identifying constraint violations together with involved entities by replacing constraints with rules \\
-- -- these rules have additional debugging-specific meta-level predicates in their heads which are instantiated when a constraint body evaluates to true; this way the debugging information is derived by datalog rules and can be queried for \\
-- -- the replacements of constraints is included in the transformation pipeline $\tau$ as an additional step \\
\begin{displaymath}
    \tau = \transdlog \circ \translt \circ \transnorm \circ \transdebug \circ \transax
\end{displaymath}
-- -- the additional transformation step $\transdebug$ is applied after conceptual syntax has been resolved, replacing constraints on the level of WSML logical expressions \\
-- -- the detailed constrained replacement performed by \transdebug can be seen from Table \ref{tab:debugging} \\
-- -- the body variables are supposed to match the appropriate form of constraint body; notice: the semantics of ofType is encoded in the meta-level axioms \mlaxioms, so ofType-constraints can't be as easily replaced but have to be generated by \transdebug \\
-- -- to maintain the constraint-semantics, some additional debugging-specific meta-level axioms, denoted by \debugaxioms, have to be included, which are shown in Table \ref{tab:debugging-axioms} \\
-- -- thus, the datalog program used for reasoning with the original WSML ontology turns to: \\
\begin{displaymath}
    P_O = \mlaxioms \cup \debugaxioms \cup \tau(O)
\end{displaymath}
-- -- then one can ask for occurrences of the different kinds of constraint violation by e.g. \\
\begin{displaymath}
    \{(a,T,I,V) : (P_O , ?\pvotype(a,T,I,V)) \rightarrow \top \}
\end{displaymath}
-- -- which asks for all occurrences of type violations by means of datalog querying mechanisms; if this set is empty then there is no problem concerning types \\

\begin{table}[tb]\label{tab:debugging}\centering
\begin{footnotesize}
\begin{tabular}{|l|l|}
  \hline
  \rule{0cm}{3.2mm} {\normalsize \emph{Constraint}} & {\normalsize \emph{Rule}} \\
  \hline
  $\transdebug(C[a$ \wsml{ofType} $T.)$ & $\pvotype(a,T,I,V)$\wsml{\lprl} \\
  & $\;C[a$ \wsml{ofType} $T]$ \wsml{and} $I$ \wsml{memberOf} $C$ \\
  & $\;I[a$ \wsml{hasValue} $V]$ \wsml{and naf} $V$\wsml{memberOf} $T.$ \\
  $\transdebug($\wsml{\cstr}$B_{mincard}.)$ & $\pvmincard(a,I)$\wsml{\lprl}$B_{mincard}.$ \\
  $\transdebug($\wsml{\cstr}$B_{maxcard}.)$ & $\pvmaxcard(a,I)$\wsml{\lprl}$B_{maxcard}.$ \\
  $\transdebug($\wsml{\cstr}$B_{user}.)$ & $\pvuser(\axiomid)$\wsml{\lprl}$B_{user}.$ \\
  \hline
\end{tabular}
\end{footnotesize}
\caption{Replacing constraints by rules for debugging.}
\end{table}

\begin{table}[tb]\label{tab:debugging-axioms}\centering
\begin{small}
\begin{tabular}{|ll|}
  \hline
  \multicolumn{2}{|l|}{\rule{0cm}{3.2mm}{\normalsize \emph{Debugging Meta-Level Axioms}}} \\
  \hline
  (1) & $\dlogcstr \pvotype(a,T,I,V)$ \\
  (2) & $\dlogcstr \pvmincard(a,I)$ \\
  (3) & $\dlogcstr \pvmaxcard(a,I)$ \\
  (4) & $\dlogcstr \pvuser(\axiomid)$ \\
 \hline
\end{tabular}
\end{small} \caption{Meta-level axioms for WSML semantics in
datalog.}
\end{table}
