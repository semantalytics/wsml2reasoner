\def\dataaxioms{\ensuremath{P_{\smtxtit{data}}}\xspace}
\def\transdpred{\transtxt{dpred}}

\subsection{Realising Datatype Reasoning}
\label{sec:datatype_reasoning}
As a result
of the transformations described previously, a generic Datalog
program is created. Although most of the generic Datalog rules are understood by practically any
Datalog implementation, realizing datatype reasoning has some
intricate challenges.

The main challenge in implementing datatype reasoning is caused by Axiom (4) in Table~\ref{tab:meta-level}, which checks the WSML \synkw{ofType} constraints. The problem is the \[\dlognot \pmo(V,C_2)\] part of the axiom, because for datatype values no explicit membership facts are included in the ontology. E.g., considering the instance \syn{MSNDialup} from the WSML ontology in Section~\ref{sec:wsml}, there is no $\pmo(10,\syn{\_integer})$ fact for the value of the \syn{providesBandwidth} attribute. I.e., any time a value is defined for an attribute constrained by an \synkw{ofType} constraint, Axiom (4) would cause a constraint violation.

To solve this problem, \pmo facts should be generated for all datatype constants that
appear in the ontology. I.e., for each
constant in the ontology axioms of the following form should
appear:
\begin{displaymath}
    \pmo(V,D) \lprl \typeof(V, D_T)
\end{displaymath} where $D$ denotes the WSML datatype, $D_T$ denotes a datatype supported by the underlying Datalog implementation, which is compatible with the WSML datatype, and \typeof denotes a built-in predicate implemented by the Datalog tool, which checks whether a constant value belongs to the specified datatype.

Those tool-specific axioms result in a new set of Datalog rules, denoted as \dataaxioms. 
I.e., the Datalog program $P_O$ that is executed by the Datalog reasoner, should be extended with this new set of rules. 
\begin{displaymath}
    P_O = \mlaxioms \cup \dataaxioms \cup \tau(O)
\end{displaymath}

In addition to datatypes, WSML also supports some built-in predicates on datatypes, such as
numeric comparison\footnote{A full list of WSML datatypes can be
found in the WSML specification \cite{wsml-spec}.}. E.g., the
definition of the \syn{SharePriceFeed\_requires\_bandwidth} axiom from the WSML ontology in Section~\ref{sec:wsml} uses a shortcut of the WSML \synkw{numericLessThan} predicate. Clearly, these special WSML
predicates have to be translated to the corresponding built-in
predicates supported by the built-in Datalog reasoner. I.e., a new, tool-specific
mapping $\transdpred : \P \mapsto \P$ has to be applied to the generic Datalog program, which translates all built-in WSML predicates into tool-specific predicates. I.e., the transformation pipeline $\tau$ is redefined as
\begin{displaymath}
    \tau = \transdpred \circ \transdlog \circ \translt \circ \transnorm \circ \transax
\end{displaymath}

To summarize the discussion, the underlying Datalog implementation
must fulfill the following requirements to support WSML datatype
reasoning:
\begin{itemize}
    \item It should provide built-in datatypes that correspond to WSML built-in datatypes.
    \item It should provide a predicate (or predicates) for checking the datatype of a constant.
    \item It should provide built-in predicates that correspond to WSML built-in predicates.
\end{itemize}

