\def\dataaxioms{\ensuremath{P_{\smtxtit{data}}}\xspace}
\def\transdpred{\transtxt{dpred}}

\subsection{Realising Datatype Reasoning}
\label{sec:datatype_reasoning} Although most of the generic
Datalog rules are understood by practically any Datalog
implementation, realizing datatype reasoning has some intricate
challenges.

The main challenge in implementing datatype reasoning is realted
to Axiom (4) in Table~\ref{tab:meta-level}, which checks attribute
type constraints. The crucial part of the axiom is the literal
\[\dlognot \pmo(V,C_2)\] because for datatype values no explicit
membership facts are included in the ontology that could
instantiate this literal. Consider, for example, the instance
\syn{MSNDialup} from the WSML ontology in Section~\ref{sec:wsml}
-- there is no fact $\pmo(10,\syn{\_integer})$ for the value of
the \syn{providesBandwidth} attribute. Any time a value is defined
for an attribute constrained by \synkw{ofType}, Axiom (4) would
cause a constraint violation.

To solve this problem, \pmo facts should be generated for all
datatype constants that appear in the ontology that could
instantiate this literal. I.e., for each constant in the ontology
axioms of the following form should appear:
\begin{displaymath}
    \pmo(V,D) \lprl \typeof(V, D_T)
\end{displaymath}
where $D$ denotes the WSML datatype, $D_T$ denotes a datatype
supported by the underlying Datalog implementation, which is
compatible with the WSML datatype, and \typeof denotes a built-in
predicate implemented by the Datalog tool, which checks whether a
constant value belongs to the specified datatype.

These additional meta-level axioms result in a new set of Datalog
rules, denoted by \dataaxioms, which are no longer in generic
datalog but use tool-specific built-in predicates of the
underlying inference engine. The Datalog program $P_O$ is extended
with this new set of rules as follows.
\begin{displaymath}
    P_O = \mlaxioms \cup \dataaxioms \cup \tau(O)
\end{displaymath}

In addition to datatypes, WSML also supports some predefined
predicates on datatypes, such as numeric comparison\footnote{A
full list of WSML datatypes can be found in the WSML specification
\cite{wsml-spec}.}. For example, the definition of the
\syn{SharePriceFeed\_requires\_bandwidth} axiom from the WSML
ontology in Section~\ref{sec:wsml} uses a shortcut of the WSML
\synkw{numericLessThan} predicate. Clearly, these special WSML
predicates have to be translated to the corresponding built-in
predicates supported by the underlying Datalog reasoner.
Therefore, we introduce a new tool-specific transformation step
\transdpred as a mapping $\P \mapsto \P$, which translates all
predefined WSML predicates in the generic Datalog program to
tool-specific built-in predicates. The transformation pipeline
$\tau$ is augmented by this additional step and is redefined as
follows.
\begin{displaymath}
    \tau = \transdpred \circ \transdlog \circ \translt \circ \transnorm \circ \transax
\end{displaymath}

To summarize the discussion, the underlying Datalog implementation
must fulfill the following requirements to support WSML datatype
reasoning:
\begin{itemize}
    \item It should provide built-in datatypes that correspond to WSML datatypes.
    \item It should provide a predicate (or predicates) for checking whether a datatype covers a constant.
    \item It should provide built-in predicates that correspond to datatype-related predefined predicates in WSML.
\end{itemize}
