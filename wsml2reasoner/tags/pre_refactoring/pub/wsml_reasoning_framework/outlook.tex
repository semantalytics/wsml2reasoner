\section{Conclusion \& Outlook\label{sec:outlook}}
We have presented a framework for reasoning in rule-based WSML
that builds on a mapping to Datalog and on querying a generic
Datalog layer. The single well-defined transformation steps can be
reused across various adaptations for different scenarios in a
highly modular way. We have incorporated debugging features by
replacing native constraints with rules to derive
debugging-relevant information that can be queried by an ontology
engineer. We have implemented our framework with two existing
reasoner tools, namely KAON2 and MINS, as alternative
implementations of the generic Datalog layer, by which we provide
the first available reasoning system for the WSML language.

While the current framework focuses on WSML-Core, -Flight and -Rule,
efforts are ongoing to extend the transformations to disjunctive
Datalog and description logics. The KAON2 system natively supports
disjunctive Datalog and DL reasoning, the latter even extended by
WSML-Flight-like rules. Also the DLV system~\cite{citrigno97dlv}
(implementing disjunctive Datalog under the stable model semantics)
can be used to realise a similar reasoning. Furthermore, we plan to
integrate the KRHyper system~\cite{wernhard03system}, which allows
reasoning with disjunctive logic programs with stratified default
negation. Transformations to DL additionally allow to incorporate
description logic system APIs to support efficient reasoning with
WSML-DL.\\[2mm]
{\bfseries Acknowledgements.} We would like to thank our
colleagues that wrote~\cite{wsml-spec}, especially Jos de Bruijn,
for fruitful discussions and the contribution of mapping
definitions.
