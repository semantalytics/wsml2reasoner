\def\dataaxioms{\ensuremath{P_{\smtxtit{data}}}\xspace}
\def\transdpred{\transtxt{dpred}}

\subsection{Realising Datatype Reasoning}
\label{sec:datatype_reasoning} Although most of the generic
Datalog rules are understood by practically any Datalog
implementation, realizing datatype reasoning has some intricate
challenges. The main challenge is related to Axiom (4) in
Figure~\ref{tab:meta-level}, which checks attribute type
constraints. The crucial part of the axiom is the literal
\[\dlognot \pmo(V,C_2)\] because for datatype values no explicit
membership facts are included in the ontology that could
instantiate this literal. Consider, for example, the instance
\wsmlname{MSNDialup} from the WSML ontology in
Section~\ref{sec:wsml} -- there is no fact
$\pmo(10,\wsmlname{\_integer})$ for the value of the
\wsmlname{providesBandwidth} attribute. Whenever a value is
defined for an attribute constrained by \wsml{ofType}, Axiom (4)
would cause a constraint violation.

To solve this problem, \pmo facts should be generated for all
datatype constants that appear as values of attributes having
\wsml{ofType} constraints in the ontology. I.e., for each such
constant in the ontology, axioms of the following form should
appear, \compress
\begin{displaymath}
    \pmo(V,D) \lprl \typeof(V, D_T) \compress
\end{displaymath}
where $D$ denotes the WSML datatype, $D_T$ denotes a datatype
supported by the underlying Datalog implementation, which is
compatible with the WSML datatype, and \typeof denotes a built-in
predicate implemented by the Datalog tool, which checks whether a
constant value belongs to the specified datatype.

These additional meta-level axioms result in a new set of Datalog
rules, denoted by \dataaxioms, which are no longer in generic
Datalog but use tool-specific built-in predicates of the
underlying inference engine. The program $P_O$ is extended by
these rules as follows. %\compress
\begin{displaymath}
    P_O = \mlaxioms \cup \dataaxioms \cup \tau(O)
\end{displaymath}

In addition to datatypes, WSML also supports some predefined
datatype predicates, such as numeric comparison
(see~\cite{wsml-spec} for a full list). The definition of the
axiom \wsmlname{SharePriceFeed\_requires\_bandwidth} from the WSML
ontology in Section~\ref{sec:wsml}, for example, uses a shortcut
of the WSML \wsml{numericLessThan} predicate (denoted by $<$). For
translation of these special predicates to the corresponding
tool-specific built-in predicates supported by the underlying
Datalog reasoner, we introduce a new tool-specific transformation
step \transdpred as a mapping $\P \rightarrow \P$. This affects
the transformation pipeline $\tau$ as follows. \compress
\begin{displaymath}
    \tau = \transdpred \circ \transdlog \circ \translt \circ \transnorm \circ
    \transax \compress
\end{displaymath}

In summary, the underlying Datalog implementation must fulfill the
following requirements to support WSML datatype reasoning: (i) It
should provide built-in datatypes that correspond to WSML
datatypes. (ii) It should provide a predicate (or predicates) for
checking whether a datatype covers a constant and (iii) It should
provide built-in predicates that correspond to datatype-related
predefined predicates in WSML.

%\begin{itemize}
%    \item It should provide built-in datatypes that correspond to WSML datatypes.
%    \item It should provide a predicate (or predicates) for checking whether a datatype covers a constant.
%    \item It should provide built-in predicates that correspond to datatype-related predefined predicates in WSML.
%\end{itemize}
